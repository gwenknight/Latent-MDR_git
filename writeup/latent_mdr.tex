\documentclass{article}
\usepackage{amsmath}
\usepackage[english]{babel}
\usepackage{breakcites}
\usepackage[font=small,labelfont=bf]{caption}
\usepackage{fullpage}
\usepackage[]{geometry}
\usepackage[]{graphicx}
\usepackage{palatino}
\usepackage{rotating}
\usepackage{subcaption}
\usepackage{tikz}
\usepackage{csvsimple}
\usepackage{multirow}
\usepackage{verbatim}
\bibliographystyle{unsrt}

\title{What has stopped MDR-TB sweeping to dominance? a fitness cost or protective latency effects?}

\author{Gwen, Rein, Pete, (Louis?)} 

\date{}

\begin{document}
 
\maketitle{}

\tableofcontents{}

\clearpage

\section{Abstract / Background / Introduction}

Rough thoughts... 

\begin{itemize}
	\item MDR-TB is bad... how long around / treatment etc
	\item Prevalence of MDR-TB at ~ 5% globally despite lots of TB treatment 
	\item Fitness cost detected in vitro, epidemiologically (clusters) and in situ (my Peru work). Perhaps this stops spread? 
	\item OR Protective effect of DS-TB carriage. Immunity. Perhaps this stops it? 
	\item Explored changing demography, latency assumptions and fitness cost levels
	\item Removed latency protection, allowing for coinfection or just remove? and compare levels of MDR-TB
	\item Also, explored impact of reduced DS-TB LTBI due to LTBI therapy - does this allow MDR to dominate? 
\end{itemize}

\clearpage

\section{Questions / Thoughts}

\begin{itemize}
	\item Should we include a fitness cost distribution? (as in my previous work). This would increase our predicted levels of MDR-TB. Could include by exploring a mean and variance combination. ut I don't think it helps to answer the theoretical question here
	\item Should we include strain immunity (Basu, 2008 work)? 
	\item Should we include mixed infections to explore immnunity / interaction effects? started to build but many unknowns. Key thing is to capture that treatment of MDR-TB can happen at a higher rate than the stats on confirmed MDR-TB receiving treatment. Could include cross-immunity. As in not just vary protection from latency, but also let DS LTBI be less likely to become DR? kinda include this with the fitness cost to transmission... 
\end{itemize}

\clearpage

\section{Methods}

\subsection{Overview}

\subsection{Model}

We built a deterministic transmission model using a standard natural history framework with two strains (DS-TB and MDR-TB). 
Uninfected, go to Latent, got to active disease. 
Treatment slightly different: track time on treatment, with fail / cure outcomes are end of each timestep. 
Previous treatment (any) is tracked in order to match statistics on levels of MDR-TB
Fitness cost applied to transmission or progression
Treatment of MDR-TB higher than just CDR as misdiagnosis / think they are DS-TB: reason for higher level of MDR in re-treatment. Use "case detection of TB" 

\subsection{Equations}

There are 7 basic subpopulations, stratified by naive and previously treated (with subscript "p"). The natural history states are uninfected ($U$), latently infected with drug susceptible ($L_S$) or MDR-TB ($L_R$), active disease with drug susceptible ($A_S$) or MDR-TB ($A_R$). There are also two treatment states: on treatment and previously infected with DS-TB ($T_S$) or with MDR-TB ($T_R$). We assumed that the false positive rate for those uninfected with \textit{M.tb.} was negligible (i.e. we did not have a previously treated uninfected sub-population).  

We modelled the population as a discrete time, single age stratified population. People could be aged 0 - 100 ($j$). Those uninfected come from new births. 
\begin{align}
U[i,1] = b 
\end{align}
where $b = birth_{rate} * p_{size}[i-1] * dt$, the product of the annual birth rate (from UN data), the population size in the previous time step ($p_{size}[i-1]$) and the timestep (dt). 

The force of infection at time step $i$ ($\lambda_S[i]$,$\lambda_R[i]$) for DS-TB or MDR-TB respectively is calculated as the rate of transmission ($\beta$) divided by the population size ($p_size$), multiplied by the number with active TB (naive and previously treated).  

\begin{align}
\lambda_S[i-1] &= \frac{\beta}{p_{size}[i-1]} \sum_{j}{p_i (A_S[i-1] + A_{Sp}[i-1])} \\
\lambda_R[i-1] &= \frac{\beta}{p_{size}[i-1]} \sum_{j}{p_i (A_R[i-1] + A_{Rp}[i-1])} 
\end{align}

Those who are uninfected transition to carrying \textit{M.tb.}. The background death rate is $m$ which varies by age as determined by UN population data. 
\begin{align}
U[i,2:100] = U[i-1,1:99] - (m[1:99]+\lambda_S[i-1]+\lambda_R[i-1])U[i-1,1:99]
\end{align}

Those with latent \textit{M.tb} infection can progress to active disease at a rate $\sigma$, or be reinfected. A proportion $p$ of all those that are (re-)infected progress immediately to active TB disease. The protective effect of being latently infected is $x$. 

\begin{equation}
\begin{split}
L_S[i,2:100] &= L_S[i-1,1:99] + \lambda_S[i-1](1 - p)(U[i-1,1:99] + xL_R[i-1,1:99]) \\ &- (\sigma + m[1:99] + x(\lambda_S[i-1]p + \lambda_R[i-1]) )L_S[i-1,1:99] \\
L_R[i,2:100] &= L_R[i-1,1:99] \\ &+ \lambda_R[i-1](1 - p)(U[i-1,1:99] + xL_S[i-1,1:99]) \\ &- (\sigma + m[1:99] + x(\lambda_R[i-1]p + \lambda_S[i-1]))L_R[i-1,1:99]
\end{split}
\end{equation}

Those with active TB disease can be detected and receive treatment at an aggregated rate $w_s[i], w_r[i]$ for DS-TB or MDR-TB respectively. There is a additional mortality rate for those with active TB ($m_a$) which is not age dependent. 

\begin{equation}
\begin{split}
A_R[i,2:100] &= A_R[i-1,1:99] + \lambda_R[i-1]p( U[i-1,1:99] + x(L_S[i-1,1:99] + L_R[i-1,1:99]) ) \\ &+ \sigma L_R[i-1,1:99] 
     - (w_r[i-1] + m[1:99
] + m_a) A_R[i-1,1:99] \\
A_S[i,2:100] &= A_S[i-1,1:99] + \lambda_S[i-1] p ( U[i-1,1:99] + x(L_S[i-1,1:99] + L_R[i-1,1:99]) ) \\ &+ \sigma  L_S[i-1,1:99] 
     - (w_s[i-1] + m[1:99
] + m_a) A_S[i-1,1:99] 
\end{split}
\end{equation} 

Those in the previously treated stratification transition through the treatment sub-populations, but otherwise have the same natural history dynamics. For those latently infected and previously treated the dynamics are:

\begin{equation}
	\begin{split}
L_{Sp}[i,2:100] &= L_{Sp}[i-1,1:99] + \lambda_S[i-1](1 - p)(xL_{Rp}[i-1,1:99]) \\ & + new_LS_from_rx + new_LS_from_rx_p \\ & - (\sigma + m[1:99] + x(\lambda_S[i-1]p + \lambda_R[i-1]) )L_{Sp}[i-1,1:99] \\
L_{Rp}[i,2:100] &= L_{Rp}[i-1,1:99] + \lambda_R[i-1](1 - p)(xL_{Sp}[i-1,1:99]) \\ & + new_LR_from_rx + new_LR_from_rx_p \\ & - (\sigma + m[1:99] + x(\lambda_R[i-1]p + \lambda_S[i-1]) )L_{Rp}[i-1,1:99]
\end{split}
\end{equation}









% \bibliography{../flusurvey}

\end{document}