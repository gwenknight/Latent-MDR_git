\documentclass{article}
\usepackage{amsmath}
\usepackage[english]{babel}
\usepackage{breakcites}
\usepackage[font=small,labelfont=bf]{caption}
\usepackage{fullpage}
\usepackage[]{geometry}
\usepackage[]{graphicx}
\usepackage{palatino}
\usepackage{rotating}
\usepackage{subcaption}
\usepackage{tikz}
\usepackage{csvsimple}
\usepackage{multirow}
\usepackage{verbatim}
\bibliographystyle{unsrt}

\title{What has stopped MDR-TB sweeping to dominance? a fitness cost or protective latency effects?}

\author{Gwen, Rein, Pete, (Louis?)} 

\date{}

\begin{document}
 
\maketitle{}

\tableofcontents{}

\clearpage

\section{Abstract / Background / Introduction}

Rough thoughts... 

\begin{itemize}
	\item MDR-TB is bad... how long around / treatment etc
	\item Prevalence of MDR-TB at ~ 5% globally despite lots of TB treatment 
	\item Fitness cost detected in vitro, epidemiologically (clusters) and in situ (my Peru work). Perhaps this stops spread? 
	\item OR Protective effect of DS-TB carriage. Immunity. Perhaps this stops it? 
	\item Explored changing demography, latency assumptions and fitness cost levels
	\item Removed latency protection, allowing for coinfection or just remove? and compare levels of MDR-TB
	\item Also, explored impact of reduced DS-TB LTBI due to LTBI therapy - does this allow MDR to dominate? 
\end{itemize}

\clearpage

\section{Questions / Thoughts}

\begin{itemize}
	\item Should we include a fitness cost distribution? (as in my previous work). This would increase our predicted levels of MDR-TB. Could include by exploring a mean and variance combination. ut I don't think it helps to answer the theoretical question here
	\item Should we include strain immunity (Basu, 2008 work)? 
	\item Should we include mixed infections to explore immnunity / interaction effects? started to build but many unknowns. Key thing is to capture that treatment of MDR-TB can happen at a higher rate than the stats on confirmed MDR-TB receiving treatment. Could include cross-immunity. As in not just vary protection from latency, but also let DS LTBI be less likely to become DR? kinda include this with the fitness cost to transmission... 
\end{itemize}

\clearpage

\section{Methods}

\subsection{Overview}

\subsection{Model}

We built a deterministic transmission model using a standard natural history framework with two strains (DS-TB and MDR-TB). 
Uninfected, go to Latent, got to active disease. 
Treatment slightly different: track time on treatment, with fail / cure outcomes are end of each timestep. 
Previous treatment (any) is tracked in order to match statistics on levels of MDR-TB
Fitness cost applied to transmission or progression
Treatment of MDR-TB higher than just CDR as misdiagnosis / think they are DS-TB: reason for higher level of MDR in re-treatment. Use "case detection of TB" 













\clearpage 

\subsection{FLU SKELETON}


\begin{figure}[htbp]
	\centering
	\includegraphics[width=1\textwidth]{../plots/compare_seasons.pdf}
	\caption{Comparing baseline data across seasons}
	\label{fig:compareseasons}
\end{figure} 

\begin{itemize}
\item don't include season - not much difference and too little data
\end{itemize}


\begin{figure}[htbp]
\centering
\begin{subfigure}[b]{0.4\textwidth}
		\includegraphics[width=1\textwidth]{../plots/ili.pdf}
		\caption{}
		\label{fig:ili}
\end{subfigure}
\begin{subfigure}[b]{0.4\textwidth}
    \includegraphics[width=1\textwidth]{../plots/ili_fever.pdf}
	\caption{}
	\label{fig:ili.fever}
\end{subfigure}
\caption{Antibiotic prescriptions split by those with ECDC standards defined Influenza like illness (ILI) (a) or an ILI and fever (b)}
\label{fig:ilif}
\end{figure}

\begin{itemize}
\item don't include ili
\item include ili.fever (SF: tracks 'flu season well)
\end{itemize}




% \bibliography{../flusurvey}

\end{document}